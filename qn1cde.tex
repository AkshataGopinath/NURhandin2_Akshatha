\section{Question 1 c,d,e}

Code for 1c, 1d and 1e:
\lstinputlisting{q1cde.py}

\subsection{1c}

Figures for KS test 1(c) are:
\begin{figure}[H]
  \centering
  \includegraphics[width=0.9\linewidth]{ks_d.png}
  \caption{1(c)}
  \label{fig:fig1}
\end{figure}

\begin{figure}[H]
  \centering
  \includegraphics[width=0.9\linewidth]{ks_pval.png}
  \caption{1(c)}
  \label{fig:fig1}
\end{figure}

%%%%%%%%%%%%%%%%%
\subsection{1d}

Kuiper's test figures for 1(d):
KS test d-statistic: it is seen that as number of random numbers sampled increases, the sampled numbers are more and more gaussian in distribution.

\begin{figure}[H]
  \centering
  \includegraphics[width=0.9\linewidth]{kuip_d.png}
  \caption{1(c)}
  \label{fig:fig1}
\end{figure}

Kuiper test d-statistic: the distance statistic calculated by me seems to be slightly higher than that calculated by astropy's kuiper's test, hence the p-values are slightly lower than expected,

\begin{figure}[H]
  \centering
  \includegraphics[width=0.9\linewidth]{kuiper_pval.png}
  \caption{1(c)}
  \label{fig:fig1}
\end{figure}
%%%%%%%%%%%%%%%%
\subsection{1e}

The results of 1(e), to check which of the random number sequences match a gaussian distribution are:

\begin{figure}[H]
  \centering
  \includegraphics[width=0.9\linewidth]{ks_data.png}
  \caption{1(c)}
  \label{fig:fig1}
\end{figure}

\begin{figure}[H]
  \centering
  \includegraphics[width=0.9\linewidth]{ksd_data.png}
  \caption{1(c)}
  \label{fig:fig1}
\end{figure}

KS test between the generated gaussian random numbers and the 10 datasets provided show that the p-value is high for dataset-3. So random number array 3 is most likely to be consistent with gussian random numbers with sigma=1 and mu=0.
Looking at the d-values, one can infer that dataset numbers 1, 4, 8 are very dissimilar to a gaussain distribution of mu=0 and sigma=1.